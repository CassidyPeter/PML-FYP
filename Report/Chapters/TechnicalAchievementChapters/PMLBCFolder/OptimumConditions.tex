\subsubsection{Optimum Conditions} \label{OptimumCondSection}

Though this project aims to numerically find the optimum conditions for the PML absorbing boundary condition, there exists a paper \cite{choung2018nonreflective} (as described in Section \ref{BackgroundSection}) which mathematically determines the optimum conditions by assuming that there are two factors inducing error within the domain: spurious waves from the lack of continuous resolution in the PML zone, and spurious waves from too little damping in the PML zone. The other 3 papers \cite{margengo1999optimumpml} \cite{li2003optimizationPML} \cite{agrawal2004pmlperformance} fail to link the relationship between PML width and PML damping coefficient via numerical methods, and so this project will be the latest PML performance research concerned with aeroacoustics.

\textcite{choung2018nonreflective} conclude that the analytical optimum width and damping coefficient range varies between damping profiles. For example, a square profile can achieve an optimum width of $D_{\mathrm{opt}} 13 \Delta x$ for a damping coefficient range of $1.30 \leq \sigma \leq 1.32$. Optimum being the smallest possible width to increase computational efficiency. The later results section will aim to assess these given optimums.
